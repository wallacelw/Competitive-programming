\chapter{Graph}

\section{Fundamentals}

\section{Network flow}

\section{Matching}

\section{DFS algorithms}

\section{Coloring}

\section{Undirected Graph}

	Bridges and Articulation Points are concepts for undirected graphs!
	
	\subsection{Bridges (Cut Edges)}

		Also called \textbf{isthmus} or \textbf{cut arc}.
		
		A back-edge is never a bridge!
		
		A \textbf{lowlink} for a vertice $U$ is the closest vertice to the root reachable using only span edges and a \textit{single} back-edge, starting in the subtree of $U$.
		
		After constructing a DFS Tree, an edge (u, v) is a bridge $\iff$ there is no back-edge from $v$ (or a descendent of $v$) to $u$ (or an ancestor of $u$)
		
		To do this efficiently, it's used $tin[i]$ (entry time of node $i$) and $low[i]$ (minimum entry time considering all nodes that can be reached from node $i$).
		
		In another words, a edge (u, v) is a bridge $\iff$ the low[v] > tin[u].

		\kactlimport{bridges.cpp}

	\subsection{Bridge Tree}

	After merging \textit{vertices} of a \textbf{2-edge connected component} into single vertices, and leaving only bridges, one can generate a Bridge Tree.

	Every \textbf{2-edge connected component} has following properties:

    \begin{itemize}
		\item For each pair of vertices {A, B} inside the same component, there are at least 2 distinct paths from A to B (which may repeat vertices).
	\end{itemize}

	\kactlimport{bridgeTree.cpp}
	
	\subsection{Articulation Points} 

	One Vertice in a graph is considered a Articulation Points or Cut Vertice if its removal in the graph will generate more disconnected components

	\kactlimport{articulation.cpp}

	\subsection{Block Cut Tree}

	After merging \textit{edges} of a \textbf{2-vertex connected component} into single vertices, one can obtain a block cut tree.

	2-vertex connected components are also called as biconnected component
	
	Every bridge by itself is a biconnected component

	Each edge in the block-cut tree connects exactly an Articulation Point and a biconnected component (bipartite graph)

	Each biconnected component has the following properties:

	\begin{itemize}
		\item For each pair of edges, there is a cycle that contains both edges
		\item For each pair of vertices {A, B} inside the same connected component, there are at least 2 distinct paths from A to B (which do not repeat vertices).
	\end{itemize}

	\kactlimport{blockCutTree.cpp}
	
	\subsection{Minimum Spanning Tree}

	A minimum spanning tree (MST) or minimum weight spanning tree is a subset of the edges
	of a connected, edge-weighted undirected graph that connects all the vertices together,
	without any cycles and with the minimum possible total edge weight.
	That is, it is a spanning tree whose sum of edge weights is as small as possible.

	\kactlimport{kruskal.cpp}

	\section{Directed Graph}

	\subsection{Topological Sort}

	Sort a directed graph with no cycles (DAG) in an order which each source of an edge is visited before the sink of this edge.

	Cannot have cycles, because it would create a contradition of which vertices whould come before.

	It can be done with a DFS, appending in the reverse order of transversal. Also a stack can be used to reverse order	

	\kactlimport{toposort.cpp}

	\subsection{Kosaraju}

	A Strongly Connected Component is a maximal subgraph in which every vertex is reachable
	from any vertex inside this same subgraph.

	A important \textit{property} is that the inverted graph or transposed graph has the same SCCs
	as the original graph.

	\kactlimport{kosaraju.cpp}

	\subsection{2-SAT}
	
		SAT (Boolean satisfiability problem) is NP-Complete.

		2-SAT is a restriction of the SAT problem, in 2-SAT every clause has exactly two variables:
		$ (X_1 \vee X_2) \wedge (X_2 \vee X_3) $

		Every restriction or implication are represented in the graph as directed edges.

		The algorithm uses kosaraju to check if any ($X$ and $\neg{X}$) are in the same Strongly Connected Component 
		(which implies that the problem is impossible). 

		If it doesn't, there is at least one solution, which can be generated using the topological sort of the same kosaraju 
		(opting for the variables that appers latter in the sorted order)

		\kactlimport{2sat.cpp}

\section{Trees}
	\kactlimport{lca.cpp}
	\kactlimport{queryTree.cpp}

\section{Math}
