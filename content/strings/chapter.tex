\chapter{Strings}

\section{Z-Function}

    Suppose we are given a string \textit{s} of length \textit{n}. 
    The Z-function for this string is an array of length \textit{n} where the \textit{i-th}
    element is equal to the greatest number of characters starting from the position \textit{i}
    that coincide with the first characters of \textit{s} (the prefix of \textit{s})

    The first element of the Z-function, z[0], is generally not well defined. 
    This implementation assumes it as z[0] = 0. 
    But it can also be interpreted as z[0] = n (all characters coincide).

    Can be used to solve the following simples problems:

    \begin{itemize}
		\item Find all ocurrences of a pattern p in another string s. 
        (p + '\$' + s) (z[i] == p.size())

        \item Find all borders. A border of a string is a prefix that is also a suffix of 
        the string but not the whole string. 
        For example, the borders of abcababcab are ab and abcab. (z[8] = 2, z[5] = 5)
        (z[i] = n-i)

        \item Find all period lengths of a string. 
        A period of a string is a prefix that can be used to generate the whole string by repeating
        the prefix. The last repetition may be partial. For example, the periods of \textit{abcabca} 
        are \textbf{abc}, \textbf{abcabc} and \textbf{abcabca}.

        It works because (z[i] + i >= n) is the condition when the common characters of z[i] in addition
        to the elements already passed, exceeds or is equal to the end of the string. For example:

        \textit{abaababaab}
        z[8] = 2

        \textbf{abaababa} is the period; the remaining (z[i] characters) are a prefix of the period; 
        and when all these characters are combined, it can form the string (which has n characters).

	\end{itemize}

    \kactlimport{zfunction.cpp}