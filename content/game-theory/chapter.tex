\chapter{Game theory}

\section{Classic Game}

\begin{itemize}

\item There are n piles (heaps), each one with $x_i$ stones.

\item  Each turn, a players must remove t stones (non-zero) from a pile, turning $x_i$ into $y_i$.

\item The game ends when it's impossible to make any more moves and the player without moves left lose.

\end{itemize}

\section{Bouton's Theorem}

Let $s$ be the xor-sum value of all the piles sizes, 
a state $s=0$ is a losing position and a state $s!=0$ is a winnig position

\subsection{Proof}

All wining positions will have at least one valid move to turn the game into a losing position.

All losing positions will only have moves that turns the game into winning positions
(except the base case when there are no piles left and the player already lost)

\section{DAG Representation}

Consider all game positions or states of the game as \textbf{Vertices} of a graph

Valid moves are the transition between states, therefore, the directed \textbf{Edges} of the graph

If a state has no outgoing edges, it's a dead end and a losing state (degenerated state). 

If a state has only edges to winning states, therefore it is a losing state.

if a state has at least one edge that is a losing state, it is a winning state.

\section{Sprague-Grundy Theorem}

Let's consider a state $u$ of a two-player impartial game and let $v_i$ be the states reachable from it.

To this state, we can assign a fully equivalent game of Nim with one pile of size $x$. 
The number $x$ is called the \textbf{Grundy value or nim-value or nimber} of the state $u$.

If \textbf{all transitions} lead to a \textit{winning state},
the current state must be a \textit{losing state} with nimber 0.

If \textbf{at least one transition} lead to a \textit{losing state},
the current state must be a \textit{winning state} with nimber > 0.

The \textbf{MEX} operator satisfies both condition above and can be used to calculate the nim-value of a state:

${nimber}_u =$ MEX of all ${{nimber}_v}_i$

Viewing the game as a DAG, we can gradually calculate the Grundy values starting
from vertices without outgoing edges (nimber=0).

Note that the MEX operator \textbf{garantees} that all nim-values smaller than the considered nimber 
can be reached, which is essentialy the nim game with a single heap with pile size = nimber. 

There are only two operations that are used when considering a Sprague-Grundy game:

\subsection{Composition}

\textit{XOR operator to compose sub-games into a single composite game}

When a game is played with multiple sub-games (as nim is played with multiple piles), 
you are actually choosing one sub-game and making a valid move there 
(choosing a pile and subtracting a value from it).

The final result/winner will depend on all the sub-games played. Because you need to play all games.

To compute the final result, one can simply consider the XOR of the nimbers of all sub-games.

\subsection{Decomposition}

\textit{MEX operator to compute the nimber of a state that has multiple transitions to other states}

A state with nimber $x$ can be transitioned (decomposed) into all states with nimber $ y < x $

Nevertheless a state may reach several states, only a single one will be used during the game.
This shows the difference between \textbf{states} and \textbf{sub-games}: All sub-games must be played
by the players, but the states of a sub-game may be ignored. 

To compute the mex of a set efficiently:

\kactlimport{mex.cpp}

\section{Variations and Extensions}

\subsection{Nim with Increases}

Consider a modification of the classical nim game: a player can now add stones to a chosen pile instead of removing.

Note that this extra rule needs to have a restriction to keep the game acyclic (finite game).

\textbf{Lemma:} This move is not used in a winnig strategy and can be ignored.

\textbf{Proof:} If a player adds $t$ stones in a pile, 
the next player just needs to remove $t$ stones from this pile. 

Considering that the game is finite and this ends sooner or later.

\textbf{Example:}
If the set of possible outcomes for a state is {0, 1, 2, 7, 8, 9}. 
The nimber is 3, because the MEX is 3, which is the smallest nim-value you can't transition into 
and also you can transition to all smaller nim-values.

Note that {7, 8, 9} transitions can be ignored, because you can simply revert the play by subtracting the same amount.

\section{Misère Game}

In this version, the player who takes the last object loses.
To consider this version, simply swap the winning and losing player of the normal version.

\section{Staircase Nim}

\subsection{Description}

In Staircase Nim, there is a staircase with n steps, indexed from 0 to n-1. 
In each step, there are zero or more coins.
Two players play in turns. In his/her move, a player can choose a step ($i > 0$)
and move one or more coins to step below it (i-1).
The player who is unable to make a move lose the game.
That means the game ends when all the coins are in step 0.

\subsection{Strategy}

We can divide the steps into two types, odd steps, and even steps.

Now let's think what will happen if a player A move $x$ coins from an even step(non-zero) to an odd step.
Player B can always move these same x coins to another even position and \textbf{the state of odd positions aren't affected}

But if player A moves a coin from an odd step to an even step, similar logic won't work.
Due to the degenerated case, there is a situation when x coins are moved from stair 1 to 0,
and player B can't move these coins from stair 0 to -1 (not a valid move).

From this argument, we can agree that coins in even steps are useless,
they don't interfere to decide if a game state is winning or losing.

Therefore, the staircase nim can be visualized as a simple nim game with only the odd steps.

When stones are sent from an odd step to an even step, it is the same as removing stones from a pile in a classic nim game.

And when stones are sent from even steps to odd ones, it is the same as the increasing variation described before.

\section{Grundy's Game}

Initially there is only one pile with x stones.
Each turn, a player must divide a pile into two non-zero piles with different sizes.
The player who can't do any more moves loses.

\subsection{Degenerate (Base) States}

x = 1 (nim-val = 0) (losing)

x = 2 (nim-val = 0) (losing)

\subsection{Other States}

nim-val = MEX (all transitions)

\subsubsection{Examples}

\textbf{x = 3:}
\begin{lstlisting}
{2, 1} -> (0) xor (0) -> 0

nim-val = MEX({0}) = 1
\end{lstlisting}

\textbf{x = 4:}
\begin{lstlisting}
{3, 1} -> (1) xor (0) -> 1

nim-val = MEX({1}) = 0
\end{lstlisting}

\textbf{x = 5:}
\begin{lstlisting}
{4, 1} -> (0) xor (0) -> 0
{3, 2} -> (1) xor (0) -> 1

nim-val = MEX({0, 1}) = 2
\end{lstlisting}

\textbf{x = 6:}
\begin{lstlisting}
{5, 1} -> (2) xor (0) -> 2
{4, 2} -> (0) xor (0) -> 0
    
nim-val = MEX({0, 2}) = 1
\end{lstlisting}

\textbf{Important observation:} All nimbers for (n $\ge$ 2000) are non-zero.
(missing proof here and testing for values above 1e6).

\section{Insta-Winning States}

Classic nim game: if \textbf{all} piles become 0, you lose. (no more moves)

Modified nim game: if \textbf{any} pile becomes 0, you lose.

To adapt to this version of nim game, we create insta-winning states,
which represents states that have a transition to any empty pile (will instantly win).
Insta-winning states must have an specific nimber so they don't conflict with other nimbers when computing.
A possible solution is nimber=INF, because no other nimber will be high enough to cause conflict. 

Because of this adaptation, we can now ignore states with empty piles, and consider them with ($null value = -1$).
And the ($nimber = 0$) now represents the states that only have transitions to insta-winning states.

After this, beside winning states and losing states, we have added two new categories of states
(insta-winning and empty-pile). Notice that:

\begin{lstlisting}[language=raw]
empty-pile <- insta-winning <- nimber(0)
\end{lstlisting}

Therefore, we have returned to the classical nim game and can proceed normally.

OBS: \textit{Empty piles} (wasn't empty before) ($nimber = -1$) is different from \textit{Non-existent piles} (never existed) (nimber = 0)

Usage Example: \url{https://codeforces.com/gym/101908/problem/B}

\section{References}

\url{https://cp-algorithms.com/game_theory/sprague-grundy-nim.html}

\url{https://codeforces.com/blog/entry/66040}

\url{https://brilliant.org/wiki/nim/}