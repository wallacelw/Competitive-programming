\chapter{Mathematics}

\section{Modular Arithmetic}

	\kactlimport{modular.cpp}

	\subsection{Lucas's Theorem}

	$$ \binom{n}{m} \equiv \prod_{i=0}^k \binom{n_i}{m_i} \quad (\text{mod } p) $$ 

	For $p$ prime. $n_i$ and $m_i$ are the coefficients of the representations of $n$ and $m$ in base $p$.

	\textit{Example:}

		11 (in base p=3) = $1 \cdot 3^2 + 0 \cdot 3^1 + 2 \cdot 3^0$

		$\implies n_2 = 1, n_1 = 0, n_0 = 2$

\section{Combinatorics}

	\begin{align*}
		\binom{n}{m} =  & \frac{ n! }{ m! \cdot (n-m)! }, & 0 <= m <= n \\
						& 0, & otherwise
	\end{align*}

	\subsection{Factorial}

		\begin{center}
			\begin{tabular}{l}
				\begin{tabular}{c|c@{\ }c@{\ }c@{\ }c@{\ }c@{\ }c@{\ }c@{\ }c@{\ }c@{\ }c}
				$n$  & 1 & 2 & 3 & 4  & 5   & 6   & 7    & 8     & 9      & 10\\
				\hline
				$n!$ & 1 & 2 & 6 & 24 & 120 & 720 & 5040 & 40320 & 362880 & 3628800\\
				\end{tabular}\\
				\begin{tabular}{c|c@{\ }c@{\ }c@{\ }c@{\ }c@{\ }c@{\ }c@{\ }c@{\ }c@{\ }c}
				$n$  & 11    & 12    & 13    & 14     & 15     & 16     & 17\\
				\hline
				$n!$ & 4.0e7 & 4.8e8 & 6.2e9 & 8.7e10 & 1.3e12 & 2.1e13 & 3.6e14\\
				\end{tabular}\\
				\begin{tabular}{c|c@{\ }c@{\ }c@{\ }c@{\ }c@{\ }c@{\ }c@{\ }c@{\ }c@{\ }c}
				$n$  & 20   & 25   & 30   & 40   & 50   & 100   & 150   & 171\\
				\hline
				$n!$ & 2e18 & 2e25 & 3e32 & 8e47 & 3e64 & 9e157 & 6e262 & \scriptsize{$>$DBL\_MAX}\\
				\end{tabular}
			\end{tabular}
		\end{center}

	\subsection{Combinatorial Struct}

		\kactlimport{combinatorics.cpp}

	\subsection{Burside Lemma}

	Let $G$ be a group that acts on a set $X$. The Burnside Lemma states that the number of distinct orbits is equal to the average number of points fixed by an element of G.
    $$T = \frac{1}{|G|} \sum_{g \in G} |\texttt{fix}(g)|$$
    Where a orbit $\texttt{orb}(x)$ is defined as
    $$\texttt{orb}(x) = \{y \in X : \exists g \in G \ gx = y \}$$
    and $\texttt{fix}(g)$ is the set of elements in $X$ fixed by $g$
    $$\texttt{fix}(g) = \{x \in X : gx = x\}$$
    
    \textbf{Example:} With $k$ distinct types of beads how many distinct necklaces of size $n$ can be made? Considering that two necklaces are equal if the rotation of one gives the other.
    \begin{center}
    \includegraphics[scale=.6, keepaspectratio]{content/math/Burnside.png}
    \end{center}
	$$\frac{1}{n} \sum_{i=1}^n k^{\gcd(i, n)}$$

	\subsection{Interesting Recursion}

	$$ f(a, b) = f(a-1, b) + f(a, b-1) $$

	$$ \implies f(a, b) = \frac{(a+b)!}{a! b!} = \binom{a+b}{a} $$

	\textit{Proof:}
	
	\begin{align*}
	&f(a, b) = \frac{(a+b)!}{a! b!}  \\ 
	\implies &f(a-1, b) = \frac{(a-1+b)!}{(a-1)! b!}, f(a, b-1) = \frac{(a+b-1)!}{a! (b-1)!}  \\
	\implies &f(a-1, b) + f(a, b-1) = \frac{(a-1+b)!}{(a-1)! b!} + \frac{(a+b-1)!}{a! (b-1)!}  \\
	\implies &f(a, b) = (a+b-1)! \cdot ( \frac{1}{(a-1)!(b)!} + \frac{1}{(a)!(b-1)!} )  \\
	\implies &f(a, b) = (a+b-1)! \cdot ( \frac{a+b}{a! b!} ) \\
	\implies &f(a, b) = \frac{(a+b)!}{a! b!} = \binom{a+b}{a} 
	\end{align*}

\section{FFT}

	FFT can be used to turn a polynomial multiplication complexity to $O(N \log{N})$.

	A \textbf{convulution} is easily computed by inverting one of the vector and doing the polynomial multiplication normally.

	\kactlimport{fft-simple.cpp}