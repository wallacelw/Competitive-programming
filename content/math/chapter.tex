\chapter{Mathematics}

\section{Modular Arithmetic}

	\kactlimport{modular.cpp}

\section{Combinatorics}

	\subsection{Factorial}

		\begin{center}
			\begin{tabular}{l}
				\begin{tabular}{c|c@{\ }c@{\ }c@{\ }c@{\ }c@{\ }c@{\ }c@{\ }c@{\ }c@{\ }c}
				$n$  & 1 & 2 & 3 & 4  & 5   & 6   & 7    & 8     & 9      & 10\\
				\hline
				$n!$ & 1 & 2 & 6 & 24 & 120 & 720 & 5040 & 40320 & 362880 & 3628800\\
				\end{tabular}\\
				\begin{tabular}{c|c@{\ }c@{\ }c@{\ }c@{\ }c@{\ }c@{\ }c@{\ }c@{\ }c@{\ }c}
				$n$  & 11    & 12    & 13    & 14     & 15     & 16     & 17\\
				\hline
				$n!$ & 4.0e7 & 4.8e8 & 6.2e9 & 8.7e10 & 1.3e12 & 2.1e13 & 3.6e14\\
				\end{tabular}\\
				\begin{tabular}{c|c@{\ }c@{\ }c@{\ }c@{\ }c@{\ }c@{\ }c@{\ }c@{\ }c@{\ }c}
				$n$  & 20   & 25   & 30   & 40   & 50   & 100   & 150   & 171\\
				\hline
				$n!$ & 2e18 & 2e25 & 3e32 & 8e47 & 3e64 & 9e157 & 6e262 & \scriptsize{$>$DBL\_MAX}\\
				\end{tabular}
			\end{tabular}
		\end{center}

\subsection{Combinatorial Struct}

	\kactlimport{combinatorics.cpp}
	
	\textbf{Interesting Recursion:}

	$$ f(a, b) = f(a-1, b) + f(a, b-1) $$

	$$ \implies f(a, b) = \frac{(a+b)!}{a! b!} = comb(a+b, a) $$

	\textit{Proof:}
	
	\begin{align*}
	   &f(a, b) = \frac{(a+b)!}{a! b!}  \\ 
	\implies &f(a-1, b) = \frac{(a-1+b)!}{(a-1)! b!}, f(a, b-1) = \frac{(a+b-1)!}{a! (b-1)!}  \\
	\implies &f(a-1, b) + f(a, b-1) = \frac{(a-1+b)!}{(a-1)! b!} + \frac{(a+b-1)!}{a! (b-1)!}  \\
	\implies &f(a, b) = (a+b-1)! \cdot ( \frac{1}{(a-1)!(b)!} + \frac{1}{(a)!(b-1)!} )  \\
	\implies &f(a, b) = (a+b-1)! \cdot ( \frac{a+b}{a! b!} ) \\
	\implies &f(a, b) = \frac{(a+b)!}{a! b!} = comb(a+b, a) 
	\end{align*}

\section{FFT}

	FFT can be used to turn a polynomial multiplication complexity to $O(N \log{N})$.

	A \textbf{convulution} is easily computed by inverting one of the vector and doing the polynomial multiplication normally.

	\kactlimport{fft-simple.cpp}