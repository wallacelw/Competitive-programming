\chapter{Data structures}

\section{Ordered Set}

    Policy Based Data Structures (PBDS) from gcc compiler

    Ordered Multiset can be created using ordered\_set\textless pll\textgreater {val, idx}

    \textbf{order\_of\_key()} can search for non-existent keys!
    
    \textbf{find\_by\_order()} requires existent key and return the 0-idx position of the given value.
    Therefore, it returns the numbers of elements that are smaller than the given value;

    \kactlimport{ordered-set.cpp}

\section{Disjoint Set Union}

    There are two optional improvements:

        -Tree Balancing 
        
        -Path Compression

    If one improvement is used, 
    the time complexity will become $O(\log{N})$

    If both are used, $O(\alpha) \approx O(5)$

    \kactlimport{dsu.cpp}

\section{Segment Tree}

    Each node of the segment tree represents the cumulative value of a range.

    \textbf{Observation:} For some problems, such as range distinct values query,
    considerer offiline approach, ordering the queries by L for example. 

    \kactlimport{segRecursive.cpp}

\section{Convex Hull Trick}

    If multiple transitions of the DP can be seen as 
    first degree polynomials (lines). CHT can be used to optimized it

    Some valid functions:

    $ax + b$
    
    $cx^2 + ax + b$ 
    (ignore $cx^2$ if c is independent)

    \kactlimport{cht-dynamic.cpp}

\section{Li-chao Tree}

    Works for any type of function that has the \textbf{transcending property}:

    Given two functions f(x),g(x) of that type, 
    if f(t) is greater than/smaller than g(t) for some x=t,
    then f(x) will be greater than/smaller than g(x) for x>t.
    In other words, once f(x) “win/lose” g(x), f(x) will continue to “win/lose” g(x).

    The most common one is the line function: $ ax + b $
    